\newcommand{\discipline}{Физика (био) -- \it 2я КР}


%% Вариант №1

\newcommand{\VarzerooneProbzeroone}{Тело массой $1$ кг свободно падает на землю с высоты $20$ м. Какую среднюю мощность развивает сила тяжести за время падения тела?}


\newcommand{\VarzerooneProbzerotwo}{Максимальная высота, на которую поднимается тело массой $1$ кг, подброшенное вертикально вверх, составляет $20$ м. Найдите чему был равен импульс тела сразу же после броска.}


\newcommand{\VarzerooneProbzerothree}{Растянутая на $2$ см стальная пружина обладает потенциальной энергией упругой деформации $4$ Дж. На сколько увеличится потенциальная энергия упругой деформации при растяжении этой пружины еще на $2$ см?}


\newcommand{\VarzerooneProbzerofour}{Горизонтальная поверхность разделена на две части: гладкую и шероховатую. На границе этих частей находится небольшой кубик массой $m= 300$ г. Со стороны гладкой части на него налетает по горизонтали шар массой $M = 0.2$ кг, движущийся со скоростью $v_0 = 3$ м/с. Определите расстояние $L$, на котором кубик остановится после абсолютно упругого центрального соударения с шаром. Коэффициент трения кубика о поверхность $\mu = 0.3$.}


\newcommand{\VarzerooneProbzerofive}{Тело движется по прямой. Под действием постоянной силы $40$ Н, направленной вдоль этой прямой, импульс тела уменьшился от $200$ кг$\cdot$м/с до $120$ кг$\cdot$м/с. Сколько для этого потребовалось времени?}


\newcommand{\VarzerooneProbzerosix}{Тело массой $0.2$ кг свободно падает без начальной скорости. За некоторый промежуток времени изменение модуля импульса тела равно $8$ кг$\cdot$м/с. Чему равен этот промежуток времени? Сопротивлением воздуха можно пренебречь.}


\newcommand{\VarzerooneProbzeroseven}{Автомобиль с выключенным двигателем сняли со стояночного тормоза, и он покатился под уклон, составляющий угол $30^\circ$ к горизонту. Проехав $10$ м, он попадает на горизонтальный участок дороги. Чему равна скорость автомобиля в начале горизонтального участка дороги?}


\newcommand{\VarzerooneProbzeroeight}{Тело движется по прямой под действием постоянной силы, равной по модулю $10$ Н. Сколько времени потребуется для того, чтобы под действием этой силы импульс тела изменился на $50$ кг$\cdot$м/с?}


\newcommand{\VarzerooneProbzeronine}{Закрепленный пружинный пистолет стреляет вертикально вверх. Какой была деформация пружины $\Delta l$ перед выстрелом, если жесткость пружины $k=1000$ Н/м, а пуля массой $5$ г в результате выстрела поднялась на высоту $h=9$ м. Трением пренебречь.}


\newcommand{\VarzerooneProbonezero}{Мальчик столкнул санки с вершины горки. Высота горки $10$ м, у ее подножия скорость санок равнялась $15${м/с}. Трение санок о снег пренебрежимо мало. Какой была скорость санок сразу после толчка?}


\newcommand{\VarzerooneProboneone}{Телу массой $0.2$ кг сообщили вертикально направленную начальную скорость $10$ м/с. Пренебрегая сопротивлением воздуха, определите модуль средней мощности силы тяжести, действовавшей на тело во время подъёма до максимальной высоты.}


\newcommand{\VarzerooneProbonetwo}{Тело движется по прямой. Начальный импульс тела равен $ 60 $ кг$\cdot$м/с. Под действием постоянной силы величиной $10$ Н, направленной вдоль этой прямой, за $5$ с импульс тела уменьшился. Чему стал равен импульс тела? }


\newcommand{\VarzerooneProbonethree}{На сани, стоящие на гладком льду, с некоторой высоты прыгает человек массой $50$ кг. Проекция скорости человека на горизонтальную плоскость в момент соприкосновения с санями равна $4$ м/с. Скорость саней с человеком после прыжка составила $0.8$ м/с. Чему равна масса саней?}


\newcommand{\VarzerooneProbonefour}{Брусок массой ${{m}_{1}}=0.5 $ кг соскальзывает по наклонной плоскости с высоты $h=1$ м и, двигаясь по горизонтальной поверхности, сталкивается с неподвижным бруском массой ${{m}_{2}}=300$ г. Считая столкновение абсолютно неупругим, определите общую кинетическую энергию брусков после столкновения. Трением при движении пренебречь. Считать, что наклонная плоскость плавно переходит в горизонтальную.}


\newcommand{\VarzerooneProbonefive}{Лебедка равномерно поднимает груз массой $200$ кг на высоту $3$ м за $5$ с. Какова мощность двигателя лебедки?}


\newcommand{\VarzerooneProbonesix}{В инерциальной системе отсчёта тело массой $2$ кг движется по прямой в одном направлении под действием постоянной силы, равной $3$ Н. На сколько увеличится импульс тела за $5$ с движения?}


\newcommand{\VarzerooneProboneseven}{Масса мотоцикла ${m}_{1}=500$ кг, масса автомобиля ${m}_{2}=1$ т. Автомобиль движется со скоростью ${v}=108$ км/ч. Отношение импульса автомобиля к импульсу мотоцикла равно $1.5$. Какова скорость мотоцикла?}


\newcommand{\VarzerooneProboneeight}{Из незакреплённой пушки стреляют снарядом массой $20$ кг, который вылетает из ствола в горизонтальном направлении со скоростью $102$ м/с относительно пушки. Пушка при этом откатывается, приобретая относительно земли скорость $2$ м/с. Чему равна масса пушки, если массой сгоревшего порохового заряда можно пренебречь?}


\newcommand{\VarzerooneProbonenine}{Тележка с кирпичами катится по инерции по горизонтальным рельсам, двигаясь со скоростью $2.2$ м/с. Общая масса тележки и кирпичей равна $100$ кг. Сопротивление движению тележки пренебрежимо мало. На тележку сверху падает кирпич массой $10$ кг и остается в ней. Скорость этого кирпича в момент падения направлена вниз перпендикулярно скорости тележки. Через некоторое время в дне тележки открывается люк, через который вертикально вниз выпадает такой же кирпич. Найдите модуль скорости, с которой будет двигаться тележка после выпадания кирпича через люк.}


\newcommand{\VarzerooneProbtwozero}{Модуль импульса частицы равен $20$ кг$\cdot$м/с, а её кинетическая энергия $40$ Дж. Оперделить массу и модуль скорости частицы?}


\newcommand{\VarzerooneProbtwoone}{Кусок льда массой $2$ кг упал без начальной скорости на землю с крыши высотой $5$ м. Пренебрегая сопротивлением воздуха, определите среднюю мощность силы тяжести, действовавшей на тело во время падения.}


\newcommand{\VarzerooneProbtwotwo}{Шарик массой $200$ г падает без начальной скорости с высоты $H = 8$ м на горизонтальный пол. После отскока от пола шарик поднимается на высоту $H_1=2$ м. Найдите модуль изменения импульса в процессе отскока шарика от пола.}


\newcommand{\VarzerooneProbtwothree}{У основания шероховатой наклонной плоскости покоится маленькая шайба массой $100$ г. Шайбе сообщают импульс $0.4$ кг$\cdot$м/с в направлении вверх вдоль наклонной плоскости. После этого шайба поднимается по плоскости и останавливается. При движении шайбы выделяется количество теплоты $0.5$ Дж. На какой высоте от основания наклонной плоскости останавливается шайба?}


\newcommand{\VarzerooneProbtwofour}{Два тела движутся с одинаковой скоростью. Кинетическая энергия первого тела в $ 4 $ раза меньше кинетической энергии второго тела. Определите отношение  $m_1 / m_2 $  масс тел. }


\newcommand{\VarzerooneProbtwofive}{Под действием силы тяги в $1000$ H автомобиль движется с постоянной скоростью $72$ км/ч. Какова мощность двигателя?}


\newcommand{\VarzerooneProbtwosix}{Тележка движется по инерции по гладким горизонтальным рельсам со скоростью $4$ м/с. На тележку вертикально сверху аккуратно опускают мешочек с песком. Масса мешочка в $3$ раза больше массы тележки. Чему будет равен модуль скорости тележки с мешочком после того, как проскальзывание мешочка относительно тележки прекратится?}


\newcommand{\VarzerooneProbtwoseven}{Какую мощность развивает двигатель подъемного механизма крана, если он равномерно поднимает плиту массой $600$ кг на высоту $4$ м за $3$ с?}


\newcommand{\VarzerooneProbtwoeight}{Мальчик массой $50$ кг находится на тележке массой $50$ кг, движущейся слева направо по гладкой горизонтальной дороге со скоростью $1$ м/с. Каким станет модуль скорости тележки, если мальчик прыгнет с неё в направлении первоначальной скорости тележки со скоростью $1.5$ м/с относительно дороги?}



